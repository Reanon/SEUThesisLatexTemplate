\documentclass[algorithmlist,figurelist,tablelist,nomlist]{template/seumasterthesis}

\usepackage{multirow} % 处理跨行表格数据
\usepackage{float}
\usepackage{lipsum}
\begin{document}

%% ----------------------------------------------------------------------------
%%                                 Meta Data
%% ----------------------------------------------------------------------------
\categorynumber{TP302.7} %《中国图书资料分类法》分类法
\UDC{004.9}              %《国际十进分类法UDC》的类号
\secretlevel{公开}        % 学位论文密级分为"公开"、"内部"、"秘密"和"机密"四种
\studentid{190001}      % 学号要完整,前面的零不能省略

%% ----------------------------------------------------------------------------
%%                           Thesis Title and Spine
%% ----------------------------------------------------------------------------
\title
    {东南大学 \LaTeX 论文模板使用手册}        % 论文中文标题
    {如何优雅地撰写硕士研究生毕业论文}         % 论文中文副标题,没有可以空着;如果标题过长,也可以用副标题实现换行
    {Southeast University \LaTeX ~Thesis Template User Manual}  % 论文英文标题
    {How to Write a Master Thesis in an Elegant Way}            % 论文英文副标题,没有可以空着;如果标题过长,也可以用副标题实现换行

\spine
	% 书脊标题与副标题
    {东南大学 \rotatebox{270}{\raisebox{2.5pt}{LaTeX}} 论文模板使用手册} 
    {}                                                               

%% ----------------------------------------------------------------------------
%%                             Author and Advidor
%% ----------------------------------------------------------------------------
\author
    {知心哥哥}                        % 作者中文姓名
    {ZHI Xin-Ge-Ge}                  % 作者英文姓名,首字母大写,姓名分开,双字用「-」连接

\advisor
    {张玉健}                % 导师中文姓名
    {ZHANG Yu-Jian}        % 导师英文姓名
    {}                     % 导师职称
    
\coadvisor                 % 联合培养导师姓名,没有可以不写
    {童飞}                  % 导师中文姓名
    {TONG Fei}             % 导师英文姓名
    {A.P.}                 % 导师职称 (English), 如教授(Prof.)、副教授(A.P.)
%% ----------------------------------------------------------------------------
%%                              Thesis Defence
%% ----------------------------------------------------------------------------
\engthesistype{应用研究}            % 工程硕士论文类型
\degreetype                        % 学位类型
    {专业硕士}
    {Master of Engineering}
\major{网络空间安全}                 % 一级学科名
\submajor{网络空间安全}             % 二级学科名
\defenddate{2022年5月27日}          % 答辩日期 \today
\authorizedate{}                  % 授予学位日期,这个档案袋不需要填
\committeechair{}               % 答辩委员会主席姓名
\reviewer{}{}            % 两位论文评阅人姓名
\department                        % 学院名称
    {网络空间安全学院}
    {School of Cyber Science and Engineering}
\seuthesisthanks                % 资助信息,没有可以不写
    {本文的部分工作受国家自然基金 No. zxgg666 的支持与帮助,在此表示感谢。}

%% ----------------------------------------------------------------------------
%%                                  Cover
%% ----------------------------------------------------------------------------
% ⚠️ 可以在编写论文的时候注释掉封面,加快编译速度
\makebigcover  % 生成A3大封面
\makecover     % 生成小封面
	 
%% ----------------------------------------------------------------------------
%%                          Abstract and Contents
%% ----------------------------------------------------------------------------
\input{chapters/abstract} 
\tableofcontents          % 生成目录
\listofothers             % 生成图、表等目录,没有可以不写
 
%% ----------------------------------------------------------------------------
%%                                Main Body
%% ----------------------------------------------------------------------------
\mainmatter                    % 开始正文
\input{chapters/chapter1}      % 第一章:
\input{chapters/chapter2}      % 第二章:
\input{chapters/chapter3}      % 第三章:
\input{chapters/chapter4}      % 第四章:
\input{chapters/chapter5}      % 第五章:
\chapter{版权信息与更新记录}
\label{chp:version_license}

\section{版权信息}

本模板基于宋睿同学发布在\href{https://github.com/TouchFishPioneer/SEU-master-thesis}{SEU-master-thesis} 并在上述工作的基础上进行了微调,解决了一些自己编写代码过程中 BUG。      % 第六章:

%% ----------------------------------------------------------------------------
%%            Acknowledgement, Appendix, Bibliography and Resume
%% ----------------------------------------------------------------------------
\input{chapters/acknowledgement}    % 致谢
\thesisbib{reference}               % 生成参考文献

%% 下面一句只是用于提示 TexPad 参考文献位置,正式生成时一定要删除
% \bibliography{reference.bib} % 告诉编译器参考文献所在文件

\input{chapters/appendix}           % 附录
\input{chapters/resume}             % 作者简介

\end{document}
